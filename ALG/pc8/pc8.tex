\documentclass[12pt]{article}
\usepackage[utf8]{inputenc}
\usepackage[T5]{fontenc}
\usepackage{graphicx,a4wide,framed,amssymb,amsmath,picinpar}
\usepackage{tikz}
\usetikzlibrary{decorations,decorations.pathmorphing,shadows}


\usepackage{algorithm}
\usepackage[noend]{algorithmic}
\usepackage{eqparbox}
\renewcommand{\algorithmiccomment}[1]{\hfill\eqparbox{COMMENT}{\# \small \tt #1}}

\newcommand{\source}[1]{\begin{flushright}\emph{[#1]}\end{flushright}}

\newcommand{\MakeScribeTop}[1]{
\noindent
\begin{framed}
\noindent
 Algorithmique Avancée 2018
 \hfill
 École Centrale-Supélec
 \\[1em]
 \centerline{ \Large
#1
 }
 \\[1em]
\centerline{  \it Christoph Dürr, Nguyễn Kim Thắng}
\end{framed}
}



\begin{document}
    \MakeScribeTop{PC8 : Algorithmes randomisés}



\section{MAX 3-SAT}

\textsc{Max 3-Sat} est similaire au \textsc{3-Sat} --- il y $n$ littéraux (variables booléennes) $x_{1}, \ldots, x_{n}$
et $m$ clauses $C_{1}, \ldots, C_{m}$. Chaque clause contient 3 littéraux et la clause est satisfait si au moins un des trois littéraux
s'évalue à vrai. L'objectif est de trouver une affectation $x_{i}$ à \textsf{Vrai} ou \textsf{Faux} telle que le nombre de clauses 
satisfaites soit maximum.
Considérer l'algorithme (extrêmement) simple suivant:

\begin{quote}
	\em Affecter $x_{i}$ pour $1 \leq i \leq n$ uniformément aléatoirement à \textsf{Vrai} ou \textsf{Faux}. 
\end{quote} 

Montrer que l'espérance du nombre des clauses satisfaites est $7/8 \cdot m$.\\
\emph{Indice: Définir $X_{j} = 1$ si la clause $C_{j}$ est satisfaite et 0 sinon.}

\section{Médian}

Soit $S$ un ensemble des $n$ entiers $\{a_{1}, \ldots, a_{n}\}$.
L'élément \emph{médian} de $S$ est le $k^{\text{ième}}$ plus grand élément de $S$
où $k = \lceil n/2 \rceil$. Considérer l'algorithme $\texttt{Select}(S,k)$ suivant.

\begin{algorithm}[ht]
\begin{algorithmic}[1]  
\STATE Choisir un élément arbitraire $a_{i} \in S$.	\label{det}
\STATE $S^{+} \gets \{a_{j}: a_{j} > a_{i}\}$ et $S^{-} \gets \{a_{j}: a_{j} < a_{i}\}$.
\IF{$|S^{-}| = k - 1$}
	\RETURN $a_{i}$
\ELSE
	\IF[l'élément est dans $S^{-}$]{$|S^{-}| \geq k$}
		\RETURN $\texttt{Select}(S^{-},k)$
	\ELSE[l'élément est dans $S^{+}$]
		\RETURN $\texttt{Select}(S^{+},k - 1 - |S^{-}|)$
	\ENDIF
\ENDIF
\end{algorithmic}
\caption{Algorithme déterministe pour \textsc{Médian}.}
\label{algo:median}
\end{algorithm}

\begin{enumerate}
	\item Prouver que $\texttt{Select}(S,k)$ retourne toujours l'élément médian. Quelle est sa complexité?
	\item Modifier la ligne \ref{det} par:
		\begin{quote}
			\em  Choisir un élément $a_{i} \in S$ \underline{uniformément aléatoirement}.
		\end{quote} 
		Prouver que l'espérance du temps d'exécution de ce nouveau algorithme est $O(n)$. 
\end{enumerate}

\section{Collectionneur de coupons}
Supposons que chaque boîte de céréales contient un des $n$ différents coupons. Dès que vous avez tous les coupons de différents
types, vous allez gagner un prix. Supposons que chaque fois vous achetez une boîte de céréale, le coupon dedans est mis de fa\c cons
uniformément aléatoirement des $n$ différentes types. 
\begin{enumerate}
	\item Combien de boîtes de céréales (en espérance) devrez-vous acheter pour gagner le prix?
	\item Quelle est la probabilité de ne pas gagner le prix après avoir acheté $2n \log n$ boîtes de céréales?
\end{enumerate}

\end{document}